\chapter{Conclusions and future work}

In this thesis, we explored the prospects of detecting two significant astrophysical phenomena: binaries in eccentric orbits and those with precessing orbital planes. We have performed a search for spinning eccentric neutron star binaries. A core component of this thesis motivates the search of eccentric and precessing binaries in the future. We have made predictions for eccentric observations for networks of second and third generation observatories. Furthermore, the necessity of searching for precessing binaries is underscored by pinpointing areas in the parameter space where current methods exhibit low sensitivity to such systems. Recognizing the potential computational demands of these searches, we have introduced an innovative approach to reduce computational expenses in matched filtering.

%Observation of a clearly eccentric or precessing binary system are crucial for determining the formation history and constraining uncertainties in various astrophysical models. To date there has been only one event with strong evidence for orbital precession and none for orbital eccentricity. Current state-of-art search methods have primarily focused on non-precessing, quasi circular binaries. To tackle these issues, we have explored three different avenues. In Chapter 4, using simple extension of current search methods and inspiral only eccentric waveforms, allowed us to reliably search for eccentric neutron star binaries. No new significant candidate was found, but we put the most stringent constraints for four different binary formation scenarios and predicted the capabilities of future detectors to use eccentric observations in determining the formation channels of compact binaries.  In Chapter 5, we shifted our focus to the next signature -- orbital precession. Extension of current search pipelines for precessing system is an ongoing effort, thus, to motivate future searches we have identified regions of the NSBH parameter space that are worst affected due to simplifying search assumptions, and quantify loss in sensitivity in these regions. Finally, in Chapter 6, we have tackled a challenge common to both search scenarios which is the increased computational costs for searching exceptional events by, that restricts our capability to search for new sources in a wide parameter region. Searches with future detectors are also expected to increased computational costs which will become challenging. We demonstrated our new hierarchical matched filtering using reduced basis to drastically reduce the costs of matched filtering. Furthermore, we have shown implementing this scheme on GPU can have significant commercial benefits. 


The key takeaways from this thesis are as follows.  While current observatories require a serendipitous detection of an eccentric system, improved second generation and future third generation observatories will be able to observe binaries even from dynamical formation channels. A fraction of these binaries will have measurable eccentricities, and we predict, observation of a clearly eccentric signal would require at least two years of observation with a network of three $A^{\#}$ and at least ten days with a network of Cosmic Explorers. Current searches are losing up to 24$\%$ of sources with mass ratios $q > 6$ and up to $\sim 60\%$ of highly precessing sources $\chi_P > 0.5$. Conducting searches for novel binaries or utilizing future observatories presents significant computational challenges. Using hierarchical search methods shows a potential reduction in floating-point operations by up to $\sim 10\times$. Furthermore, using GPUs for these searches is up to 100 times more energy and cost-effective, highlighting the crucial role of GPUs in gravitational wave detection efforts.


Improvements to the second generation and the upcoming third generation observatories are expected to be operational in the early to mid 2030s. These new detectors are set to revolutionize the field by exponentially increasing the number of observed GW events from around  $\mathcal{O}(100)$ as seen with the second-generation observatories up to $\mathcal{O}(10^5)$. This expected surge in observations will necessitate considerable efforts in various areas ranging from theoretical models to data analysis techniques. This is when our search methods and waveform models should be on par to capture novel binary events, that will be able to answer some of the most awaited answers in astronomy, astrophysics and physics. 

Eccentricity and precession can exhibit similar characteristics \cite{Romero-Shaw:2022fbf}, leading to difficulties in distinguishing their respective influences on formation channels. The optimal strategy for developing waveform approximants that account for both non-zero eccentricity and precession involves creating a surrogate model based on a set of eccentric and precessing numerical relativity waveforms. However, this approach is likely several years away from being feasible. The challenge lies in the scarcity of numerical relativity waveforms that are both eccentric and precessing, coupled with the fact that generating these simulations requires considerable time and energy. In the interim, until dependable waveform models become available, our focus will remain on exploring new binary searches. We will propose three different future prospects of research based on the extension of works in this thesis below. 


Eccentric BBHs have been only searched using the less sensitive unmodeled searches. This is due to lack of fast IMR eccentric waveforms -- the merger is in the sensitive band of the current detectors and thus, a modeled eccentric search for BBHs also requires the merger and the ringdown part. Rapid development is underway to make faster, eccentric IMR waveform models that cover a wide parameter range \cite{Klein:2021jtd, Joshi:2022ocr}. With such models available, we aim to perform the first modeled eccentric BBH search using the publicly available data. The prospects of observing a clearly eccentric BBH may be slightly better due to higher detection rate for BBHs compared to neutron star. Constraints on dense star cluster will become significantly tighter with 10 eccentric BBH observations \cite{Zevin:2021rtf}.      

Efforts for developing search methods for precessing system is currently underway, involving decomposition of precessing waveforms \cite{McIsaac:2023ijd,Fairhurst:2019vut}. A search method using original precessing waveforms has to address several problems. First, is defining the parameters to be included in the template bank. Second, is to define the coincidence between any two detectors -- the condition of exact-match coincidence is no longer valid, because a precessing signal may be triggered with different template parameters in different detectors. We plan to develop a precessing search method by implementing a two-stage hierarchical scheme that will use a coarser template bank for the first-stage and then a follow-up with a rapid parameter estimation. We intend to use the component masses, the effective spin parameters, the inclination angle and an extra angle to be included in the template bank. Triggers only from the same bank will be followed up for the coincidence test. 

% 
Searches for sub-solar or low-mass binary systems provide a wealth of information on the equation of state of dense nuclear matter, dark matter or new physics \cite{Nitz:2022ltl,LIGOScientific:2022hai}. The detection of a subsolar-mass black hole would definitively confirm the existence of Primordial Black Holes (PBHs), which are candidate for dark matter \cite{Sasaki:2018dmp, Carr:2020gox}. Detecting a low-mass neutron star would provide valuable constraints on the equation of state of dense nuclear matter \cite{Silva:2016myw}. However, as previously mentioned, the search for these low-mass binaries demands substantial computational resources. Furthermore, incorporating tidal deformabilities to potentially enhance the sensitive volume by approximately $80\%$ \cite{Bandopadhyay:2022tbi}, would significantly escalate the search costs. This challenge further motivates the need for applying our hierarchical matched-filtering approach. We plan to integrate this method into the PyCBC search pipeline, enabling more cost-effective and energy-efficient execution of any modeled search.

%Search for sub-solar binaries can provide constraints on primordial black hole population, which are candidates. Sub-solar binaries can be attributed to primordial black hole population, which are candidate for dark matter \cite{}.   Sub-solar or low-mass compact objects may consist of primordial black holes (PBH)s \cite{}. PBHs are candidates for dark matter, and searching for low-mass binaries could provide further insights \cite{}.


%No pior searches for precessing system and no modeled searches for eccentric BBH systems have been performed to date. This means a large parameter space remains unexplored to look for these systems. As mentioned earlier, one of the issues in searching for these systems is the large amount of computational resources required, which can be tackled by our new matched filtering scheme. In future we plan to integrate our method into the standard PyCBC search pipeline and perform a hierarchical search for precessing systems targeting the regions of poor sensitivity identified in our study. Other limitation is the lack of fast and accurate eccentric IMR waveforms which prohibit us from searching for eccentric BBH models. However, currently a model is being developed which is expected to be accurate and fast enough for us to perform an eccentric BBH search soon.


   



%Challenges
%suffers from lack of fast, accurate eccentric waveforms and in case of precessing binaries,
