\chapter{Introduction}

\section{The Compelling Allure of Observational Astronomy}
%Some of the fundamental questions that humankind has not yet been able which leads to series of question why earth is the only habitable planet ? how our energy source the Sun was formed ? how the solar system was formed, and so on. We can keep on zooming out until we ask how everything started. Just like historians study the past, studying the systems beyond on cosmic scales can give insights into where do we come from and where are we going. 

%From the birth of astronomy with Galileo about 500 years ago by him peering into the night sky to the technological advancements has made bigger and more sensitive observatories which has allowed us to discover various celestial bodies, galaxies and the cosmos as a whole. Galileo's use of telescopes in the early 17th century revolutionized astronomy. He observed the Moon's craters, Jupiter's moons, and the phases of Venus, challenging the geocentric model and supporting the heliocentric theory proposed by Copernicus. 

%The 20th century saw the development of powerful telescopes, both on Earth and in space. The Hubble Space Telescope, launched in 1990, provided unprecedented views of distant galaxies and other cosmic phenomena, leading to groundbreaking discoveries about the expansion of the universe and the age of the cosmos. 

From the dawn of time, the cosmos has called out to humankind, stirring our innate curiosity and wonder. To study astronomy is to pursue answers to the most profound questions of our existence. Why are we here? Are we alone? What is the origin and fate of our universe? Through observing the skies, we are not only exploring the vast expanse of the universe but also embarking on a voyage of self-discovery, understanding our place in the grand cosmic story.

Astronomy's journey began with rudimentary observations of the night sky. One pivotal moment in this voyage of understanding was when Galileo, armed with an early telescope in the 17th century, gazed up and revealed the moons of Jupiter and the phases of Venus, challenging long-standing beliefs. As time progressed, our tools for exploration evolved dramatically. The 20th and 21st centuries ushered in a golden age of space-based telescopes, like the Hubble Space Telescope and the recently launched James Webb Telescope, that unfurled the universe in breathtaking clarity, ranging across the electromagnetic spectrum from radio to gamma rays. Through these observations, our understanding of the universe has expanded exponentially, unraveling the secrets of stars, galaxies, black holes, and the very fabric of spacetime itself.

While telescopes were peering into the depths of space, theoretical frameworks were being laid for an entirely new kind of astronomy—one based on gravitational waves. These are ripples in spacetime itself, first posited by Albert Einstein as a part of his General Theory of Relativity in 1915. Though the theory was revolutionary, confirming the existence of gravitational waves was a monumental challenge. The first indirect evidence came in 1974, when astronomers Russell A. Hulse and Joseph Hooton Taylor Jr. observed a binary pulsar system whose orbital decay matched predictions based on the energy lost to gravitational radiation. Their groundbreaking discovery earned them the Nobel Prize in Physics in 1993.

In the intervening years, many ambitious endeavors were undertaken to directly detect these elusive waves. Early attempts included constructing "bar detectors," large metallic cylinders designed to resonate when struck by a gravitational wave. However, these were not sensitive enough due to various instrumental challenges including thermal noise and seismic interference.

Recognizing the limitations of bar detectors, researchers shifted their focus to laser interferometry, a technique that could measure distortions in spacetime with unprecedented precision. The result was the Laser Interferometer Gravitational-Wave Observatory (LIGO), a massive L-shaped interferometer with arms stretching several kilometers. Even with such an advanced setup, detection was far from guaranteed. However, on September 14, 2015, almost a century after Einstein first posited their existence, gravitational waves from the collision of two black holes were directly detected for the first time.

This seminal event marked the birth of gravitational wave astronomy, adding a new layer of richness to our understanding of the cosmos. Just like traditional observational astronomy has provided invaluable insights into the electromagnetic behaviors of celestial bodies, gravitational wave astronomy promises to unlock secrets from phenomena that do not emit light or any other form of electromagnetic radiation. This field is still in its infancy, but the waves it has created are already reverberating through the scientific community, ushering us into a new age of cosmic exploration.

\section{Chapter descriptions}