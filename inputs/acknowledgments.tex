\chapter{Acknowledgments}

The past four years at the Max Planck Institute in Hannover have been a roller coaster ride, beginning around the Covid era. It was filled with unexpected challenges, but I was fortunate to have many individuals who supported and uplifted me throughout my PhD.

I extend my profound gratitude to Alexander Nitz for his unwavering guidance and patience, which have been crucial to my growth. His skill in challenging me, helping me grasp the intuition behind problems, and particularly in refining my soft skills, has been instrumental in my development. Being under his mentorship has been an immense privilege.

I am thankful to Bruce Allen and Badri Krishnan for granting me the extraordinary opportunity to undertake my PhD at this institute within the compact binary merger group, and for pushing me towards its completion.

I want to thank my collaborators -- Collin Capano, Sumit Kumar, Yifan Wang, Marlin Sch\"afer, Alexander Nitz, Shichao Wu, Shilpa Kastha, and Miriam Cabero. Their collective efforts in publishing two independent gravitational-wave catalogs (3-OGC and 4-OGC) played a significant role in this thesis.

I am indebted to Henning Fehrmann for his support at the start of my PhD and for the opportunity to work on his project. This experience introduced me to GPU programming, a skill that I find invaluable for my future. The ATLAS supercomputer, maintained by the ATLAS scientific team, was essential for my research. I am thankful to Carsten Aulbert and Henning for their swift technical support and for accommodating the demands of my resource-intensive scripts.      

My sincere thanks to those who proofread this thesis, which significantly improved the clarity of this thesis - Cecilio García Quirós, Alexander Nitz, Xisco J. Forteza, Sayak Dutta, Sumit Kumar, Gianluca Pagliaro, and Reid Ferguson. 

A special thanks go to Xisco Forteza and Sumit Kumar for being readily available for discussions whenever I had trouble understanding a concept or with PyCBC. Our frequent brainstorming sessions, both academic and after-work beers, have contributed enormously to my personal and academic growth.

A heartfelt thanks to Gabi, Luisa, and Tamara for guiding me through the thick bushes of German bureaucracy and for making life at the Institute easier.  I am also grateful to the Mensa group for the everyday chat over lunch, and for the immediate mocha sessions that fueled my coffee addiction. 

I want to thank my friends, who kept me in good company outside my ``working" hours and made Hannover feel like home. The list is long, and I will try my best to call out everyone. In no particular order, thanks to --  Kaushik, Gianluca, Miguel, Saruul, Jordina, Xisco, Sumit, Reid, Cameron, Alicia, Qazal, Marina, Victor, Angela, Yulia, Sebastian, Het, Kashyap, Riddhi, Neha, Brian, Rodrigo, Jasper, Giada, Matteo, Tomislav, Narjiss, Sari, Anjana, Jing, Marlin, Sayak, Lars, Serena, Prajwal, Benjamin, Prasanna, Pep, Lo, Reshma, Johny, Tom, Jorge, Caroline, Sergio, Jose, Stefano, Hossein, Saurabh, Gemma, Krishnendu, Pierre, and Cecilio. 

Finally, immense gratitude to my family - my mom, dad, and brother, for their constant support from the other side of the world. Their belief in my ability to get a PhD in physics has been a major driving force in my journey.

