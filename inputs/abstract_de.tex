\chapter*{Kurzfassung}
\vspace*{-1.cm}
Die Gravitationswellen-Astronomie (GW) hat eine neue Ära der wissenschaftlichen Erforschung eingeläutet, die uns noch nie dagewesene Einblicke in kompakte binäre Verschmelzungsquellen wie binäre Schwarze Löcher und Neutronenstern-Doppelsterne gewährt. GW-Detektoren wie Advanced LIGO und Virgo haben fast hundert verschmelzende Doppelsterne beobachtet, aber ihre Ursprünge sind nach wie vor ungeklärt. Merkmale wie Bahnexzentrizität oder Bahnpräzession sind Anzeichen für Doppelsterne, die in einer dichten Umgebung entstanden sind oder bei denen es während ihrer Entwicklung zu \\ Mehrkörperwechselwirkungen kam. Wenn solche Doppelsterne entdeckt werden, würde dies auf einen dynamischen Entstehungskanal hinweisen. Die Entdeckung solcher Systeme könnte auch Einblicke in seit langem ungewisse astrophysikalische und physikalische Prozesse wie die Entwicklung der gemeinsamen Hülle, die Geburtsstöße von Supernovae und die Dynamik von dichten Umgebungen geben. Üblicherweise wird nach Doppelsternen mit quasi-kreisförmigen Bahnen gesucht, deren Spins auf den Bahndrehimpuls ausgerichtet sind und die nur die dominante Mode der Gravitationsstrahlung einfangen. Die Lockerung einer dieser Suchannahmen erfordert eine bis zu 100-fach höhere Rechenleistung als eine äquivalente nicht-exzentrische, nicht-präzessive Suche. In dieser Arbeit stoßen wir an die Grenzen der bestehenden Suchpipelines, um nach neuartigen Doppelsternen zu suchen. Wir gehen dieses Problem in drei verschiedenen Richtungen an. Erstens, indem wir aktuelle Suchmethoden erweitern, um nach exzentrischen Systemen zu suchen - wir haben die erste Suche nach sich drehenden, exzentrischen Neutronenstern-Doppelsternen in den öffentlichen Daten der Observatorien Advanced LIGO und Virgo durchgeführt. Mit Hilfe unserer Suchergebnisse bringen wir verschiedene astrophysikalische Modelle auf den neuesten Stand der Beobachtungen und sagen spannende Ergebnisse für zukünftige Observatorien vorher. Zweitens untersuchen wir, wie viele präzidierende Quellen von den aktuellen Suchpipelines übersehen werden, und identifizieren Regionen des Parameterraums, die für gezielte Präzessionssuchen entscheidend sind. Schließlich befassen wir uns mit einem gemeinsamen Problem bei der Suche nach einem der beiden neuen Typen von Doppelsternen - den erhöhten Rechenkosten. Wir demonstrieren eine neue Optimalfilter-Technik, die bis zu $10\times$ der Rechenkosten einsparen kann, ohne an Sensitivität zu verlieren, und die auf alle modellierten Suchschemata angewendet werden kann.

\clearpage
\thispagestyle{plain}
\textbf{Schlüsselwörter:} Gravitationswellen, kompakte binäre Verschmelzungen, Suchalgorithmen für Gravitationswellen